% Created 2017-10-25 Wed 20:47
% Intended LaTeX compiler: pdflatex
\documentclass[11pt]{article}
\usepackage[utf8]{inputenc}
\usepackage[T1]{fontenc}
\usepackage{graphicx}
\usepackage{grffile}
\usepackage{longtable}
\usepackage{wrapfig}
\usepackage{rotating}
\usepackage[normalem]{ulem}
\usepackage{amsmath}
\usepackage{textcomp}
\usepackage{amssymb}
\usepackage{capt-of}
\usepackage{hyperref}
\author{Francesco Antonello Ferraro}
\date{\today}
\title{Calculadora de Pilha Encadeada}
\hypersetup{
 pdfauthor={Francesco Antonello Ferraro},
 pdftitle={Calculadora de Pilha Encadeada},
 pdfkeywords={},
 pdfsubject={},
 pdfcreator={Emacs 25.3.2 (Org mode 9.1.1)}, 
 pdflang={English}}
\begin{document}

\maketitle
\begin{abstract}

 O algoritmo da calculadora é bastente simples. Primeiramente, na classe App, lemos todas as linhas do arquivo, removemos todos os caractéres desnecessários. Cada linha lida em um arquivo chamado `exemplo.txt` gera uma ação a ser executada na classe Uno ultilizando o método público handleCommand. 
Esse método basicamente verifica se os caractéres lidos em cada linha são numéricos ou não. Em caso positivo, ele adiciona esse elemento a um pilha interna à classe Uno. Em caso negativo, ele identifica qual operador foi passado ao programa e executa a tarefa específica a cada operador nessa mesma pilha interna. A pilha interna da classe Uno foi implementada paralelamente na classe Stack ultilizando uma lista encadeada para manter as informações.

\end{abstract}




O trablho foi realizado um um ambiente Gradle que está nesse
repositório
\href{https://github.com/cescoferraro/eclim/tree/feature/hp}{https://github.com/cescoferraro/eclim/tree/feature/hp}. Para facilitar
o processo de testar o código eu adionei o arquivo jar com o resultado
do código implementado na pasta de entrega do trabalho.

\begin{verbatim}
java -jar hp.jar
\end{verbatim}

\section{Classe App}
\label{sec:org843c061}


Basicamente somente copia o código encontrado no enunciado do trabalho
daixando o algoritmo para a classe Uno.

\section{Classe Uno}
\label{sec:org6a7d657}

O método handleCommand é o coração da classe. Ele verifica se o
comando enviado é um inteiro ou uma operação conhecida pela
calculadora, executando o algoritmo pertinente a cada commando.
O método top retorna o valor do último elemento da pilha privada à
classe.  O método size retorna a quantidade de elementos na pilha
num determinado instante.

\begin{center}
\begin{tabular}{ll}
Modifier and Type & Method and Description\\
\hline
void & handleCommand(java.lang.String cmd)\\
java.lang.Double & top()\\
int & size()\\
\end{tabular}
\end{center}

\section{Classe Stack}
\label{sec:org4564644}

A classe stack implementa uma pilha utilizando um lista encadeada.

\begin{center}
\begin{tabular}{ll}
Modifier and Type & Method and Description\\
\hline
add & add(Double e)\\
void & handleCommand(java.lang.String cmd)\\
boolean & isEmpty()\\
Double & peek()\\
Double & pop()\\
Double & size()\\
\end{tabular}
\end{center}
\end{document}
