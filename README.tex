% Created 2017-10-25 Wed 17:13
% Intended LaTeX compiler: pdflatex
\documentclass[11pt]{article}
\usepackage[utf8]{inputenc}
\usepackage[T1]{fontenc}
\usepackage{graphicx}
\usepackage{grffile}
\usepackage{longtable}
\usepackage{wrapfig}
\usepackage{rotating}
\usepackage[normalem]{ulem}
\usepackage{amsmath}
\usepackage{textcomp}
\usepackage{amssymb}
\usepackage{capt-of}
\usepackage{hyperref}
\author{Francesco Antonello Ferraro}
\date{\today}
\title{Calculadore de Pilha Encadeada}
\hypersetup{
 pdfauthor={Francesco Antonello Ferraro},
 pdftitle={Calculadore de Pilha Encadeada},
 pdfkeywords={},
 pdfsubject={},
 pdfcreator={Emacs 25.3.2 (Org mode 9.1.1)}, 
 pdflang={English}}
\begin{document}

\maketitle
\begin{abstract}
 O algoritmo da calculadora é bastente simples. Primeiramente, na classe App, lemos todas as linhas do arquivo, removemos todos os caractéres desnecessários. Cada linha lida em um arquivo chamado `exemplo.txt` gera uma ação a ser executada na classe Uno ultilizando o método público handleCommand. 
Esse método basicamente verifica se os caractéres lidos em cada linha são numéricos ou não. Em caso positivo, ele adiciona esse elemento a um pilha interna à classe Uno. Em caso negativo, ele identifica qual operador foi passado ao programa e executa a tarefa específica a cada operador nessa mesma pilha interna. A pilha interna da classe Uno foi implementada paralelamente na classe Stack ultilizando uma lista encadeada para manter as informações.
\end{abstract}

\section{Clase App}
\label{sec:orga83a469}
\section{Clase Uno}
\label{sec:org27ab679}

\begin{center}
\begin{tabular}{ll}
Modifier and Type & Method and Description\\
\hline
void & handleCommand(java.lang.String cmd)\\
java.lang.Integer & top()\\
\end{tabular}
\end{center}


\section{Clase Stack}
\label{sec:org7bdfd7b}

\begin{center}
\begin{tabular}{ll}
Modifier and Type & Method and Description\\
\hline
add & add(int e)\\
void & handleCommand(java.lang.String cmd)\\
boolean & isEmpty()\\
int & peek()\\
int & pop()\\
int & size()\\
\end{tabular}
\end{center}
\end{document}
